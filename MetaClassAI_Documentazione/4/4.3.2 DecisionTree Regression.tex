\subsubsection{DecisionTree Regression}
\fancyhead[]{}
\fancyhead[R]{DecisionTree Regression}

In seguito alla realizzazione del modello utilizzante un semplice algoritmo di Regressione
Lineare, le sue prestazioni sono state confrontate con altri modelli che fanno utilizzo di diversi
algoritmi di regressione. Il primo algoritmo con cui è stato confrontato è stato il \textbf{Decision Tree}, che non si limita a predire dati con vincoli di linearità, ma permette anche di predire
valori che si presentano sotto forma di curve.\\
Riportiamo di seguito i risultati ottenuti:
\begin{table}[!htbp]
    \centering
    \caption{DecisionTree Regression}
    \begin{tabular}{|c|c|c|c|c|c|c|}
        \hline
        Test & Regressor & Alg & Scaler & MAE & MSE & RMSE \\
        \hline
        1 & DecisionTree & KF & ZScore & 18.90721 & 512.38004 & 22.62610 \\
        \hline
        2 & DecisionTree & RKF & ZScore & 18.57162 & 508.82534 & 22.54597 \\
        \hline
        3 & DecisionTree & KF & MinMax & 19.22090 & 529.42730 & 22.99655 \\
        \hline
        4 & DecisionTree & RKF & MinMax & 18.58814 & 509.21055 & 22.55407 \\
        \hline
        5 & DecisionTree & KF & Robust & 19.05921 & 518.14221 & 22.75477 \\
        \hline
        6 & DecisionTree & RKF & Robust & 18.56476 & 508.42116 & 22.53678 \\
        \hline
    \end{tabular}
    \label{tab:reference}
\end{table}

Si può facilmente notare come i risultati ottenuti siano tutti peggiori rispetto a quelli
ottenuti con l’algoritmo di regressione lineare.