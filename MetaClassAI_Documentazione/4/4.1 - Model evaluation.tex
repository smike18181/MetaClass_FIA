\subsection{Model evaluation}
\fancyhead{}    % reset header
\fancyhead[R]{Model evaluation}

In seguito alla definizione delle tecniche di ingegnerizzazione dei dati dell’agente intelligente, è necessario stabilire le \textbf{metriche} e le \textbf{tecniche} di validazione delle prestazioni dello
stesso. Occorre suddividere l’insieme dei dati fin’ora analizzato in due insiemi: 
\begin{itemize}
\item \textbf{training set}:
composto dalle istanze di dati che saranno utilizzate per l’addestramento.
\item \textbf{test set}, composto dalle istanze di dati per cui l’agente dovrà predire il valore della variabile dipendente.
\end{itemize}
Abbiamo preso in considerazione diverse tecniche per effettuare questa suddivisione: \textit{K-Fold
validation, Stratified K-fold validation, Repeated K-fold validation e Repeated Stratified
K-fold validation}. In particolare, le tecniche \textit{Stratified K-fold} e \textit{Repeated Stratified} sono state
scartate in quanto non risultano adatte ad essere applicate su dati continui e a dataset con più
variabili indipendenti. \\ 
Per tutte è stato necessario definire il valore di K, ovvero il numero
di gruppi utilizzati per suddividere il dataset in test e training set, mentre per le tecniche di tipo "Repeated" è stato stabilito anche il numero di
ripetizioni di validazione da effettuare. Le metriche che sono state impiegate per valutare la bontà delle
previsioni effettuate sono state: