\subsubsection{SVR Regression}
\fancyhead[]{}
\fancyhead[R]{SVR Regression}

Questo si basa sul più
noto algoritmo di classificazione SVM, il quale costruisce un iperpiano che utilizza per la
predizione di risultati. Come per l’algoritmo Random Forest, anche per questo abbiamo
dovuto modificare i parametri della Repeated K-Fold validation.

I risultati ottenuti sono i seguenti:
\begin{table}[!htbp]
    \centering
    \caption{SVR Regression}
    \begin{tabular}{|c|c|c|c|c|c|c|}
        \hline
        Test & Regressor & Alg & Scaler & MAE & MSE & RMSE \\
        \hline
        1 & SVR & KF & ZScore & 13.71190 & 253.85640 & 15.92749 \\
        \hline
        2 & SVR & RKF & ZScore & 13.69320 & 253.25402 & 15.91111 \\
        \hline
        3 & SVR & KF & MinMax & 13.70895 & 253.73213 & 15.92403 \\
        \hline
        4 & SVR & RKF & MinMax & 13.69007 & 253.17711 & 15.90868 \\
        \hline
        5 & SVR & KF & Robust & 13.70853 & 253.27096 & 15.90938 \\
        \hline
        6 & SVR & RKF & Robust & 13.69518 & 252.91018 & 15.90045 \\
        \hline
    \end{tabular}
    \label{tab:random_forest}
\end{table} 

Anche qui possamo notare che le metriche sono molto vicine a quelle del Lasso Regression, della Linear Regression e del Ridge Regression