\subsubsection{Linear Regression}
\fancyhead[]{}
\fancyhead[R]{Linear Regression}

Un regressore lineare, o modello di regressione lineare, cerca di modellare la relazione tra una variabile dipendente e una o più variabili indipendenti attraverso una \textbf{funzione lineare}. \\
Applicando la funzione di generazione del modello vista nel paragrafo \ref{paragrafo 4.3} sono stati ottenuti questi risultati:
\begin{table}[!htbp]
    \centering
    \caption{Linear Regression}
    \begin{tabular}{|c|c|c|c|c|c|c|}
        \hline
        Test & Regressor & Alg & Scaler & MAE & MSE & RMSE \\
        \hline
        1 & Linear & KF & ZScore & 13.61091 & 249.70188 & 15.79693 \\
        \hline
        2 & Linear & RKF & ZScore & 13.63839 & 250.71404 & 15.82796 \\
        \hline
        3 & Linear & KF & MinMax & 13.61091 & 249.70188 & 15.79693 \\
        \hline
        4 & Linear & RKF & MinMax & 13.63839 & 250.71404 & 15.82796 \\
        \hline
        5 & Linear & KF & Robust & 13.61091 & 249.70188 & 15.79693 \\
        \hline
        6 & Linear & RKF & Robust & 13.63839 & 250.71404 & 15.82796 \\
        \hline
    \end{tabular}
    \label{tab:reference}
\end{table}

Possiamo notare  come la differenza nelle prestazioni ottenute utilizzando le due modalità di validazione \textit{KF e RKF} siano
a favore per la tecnica  \textbf{K-fold}. \\
Del resto, applicando una delle 3 tecniche di scaling non si nota nessuna differenza sostanziale.