\subsubsection{Regressione non lineare}
\fancyhead{}    % reset header
\fancyhead[R]{Regressione non lineare}
\label{paragrafo 4.2.2}
I modelli di regressione lineare semplici hanno principalmente 2 limitazioni, che li
rendono difficilmente utilizzati nel concreto:
\begin{enumerate}
    \item Difficoltà nella rappresentazione di relazioni non lineari.
    \item Propensione all'overfitting con l'aumentare del numero di feature.
\end{enumerate}
Il primo problema è semplicemente risolvibile con gli alberi decisionali, per mitigare al
secondo problema occorre considerare i \textbf{gradi di libertà}, ovvero i punti in cui la funzione
tende a flettersi e a curvarsi, ciò viene risolto per esempio attraverso la \textit{Ridge Regression.}