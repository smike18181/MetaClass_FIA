\subsection{Scelta del dataset}
\fancyhead{}    % reset header
\fancyhead[R]{Scelta del dataset}
\label{paragrafo 2.1}
\par{
L'acquisizione dei dati rappresenta il primo fondamentale passo nel processo di sviluppo di un modello di intelligenza artificiale. Questo processo consiste nella raccolta, nell'organizzazione e nella preparazione dei dati necessari per addestrare e valutare il modello.\newline
Dopo un'accurata fase di ricerca, si è proceduto con la creazione del dataset necessario per il modello di intelligenza artificiale finalizzato alla stima della durata consigliata per l'utilizzo dell'esperienza VR.\newline
Tale dataset è stato ottenuto tramite la \textbf{modellazione} e \textbf {l'elaborazione} di dati preesistenti, i quali sono stati ricavati dal sito \href{https://www.kaggle.com/}{\textbf{kaggle.com}}.\newline
In particolare, il dataset scelto è stato il seguente: \href{https://www.kaggle.com/datasets/aakashjoshi123/virtual-reality-experiences/data}{\textbf{Virtual Reality Experiences}}.\newline
Questo dataset, reso disponibile con licenza pubblica, consiste in un insieme di \textbf{1000 istanze} ricavate dalle esperienze di  utenti all'interno di ambienti di realtà virtuale (VR),
comprendendo informazioni sulle risposte fisiologiche e emotive degli utilizzatori del sistema.
}