\subsection{Analisi Dataset e ei dati al suo interno}
\fancyhead{}    % reset header
\fancyhead[R]{Analisi Dataset e ei dati al suo interno}
\label{paragrafo 2.2}
\par{
Le fonti del dataset sono costituite principalmente da informazioni ricavate da studi effettuati in laboratorio.
\textbf{il Dataset si compone delle seguenti sette caratteristiche:}\newline
\begin{itemize}
  \item \textbf{User ID}
    \begin{itemize}
      \item Questa Caretteristica assegna un ID distinto a ciascun utente per differenziare i propri dati nel set di dati.
    \end{itemize}
  \item \textbf{Age}
    \begin{itemize}
      \item . Rappresenta l'età dell'utente al momento dell'esperienza VR è rappresentato da un valore intero.
    \end{itemize}
  \item \textbf{Gender}
    \begin{itemize}
      \item Questa variabile indica il sesso dell'utente. Può avere categorie come "Maschio", "Femmina" o "Altro", che rappresentano l'identità di genere dell'utente.
    \end{itemize}
  \item \textbf{VR Headset Type}
    \begin{itemize}
      \item La variabile identifica il tipo di visore VR utilizzato dall'utente durante l'esperienza VR
    \end{itemize}
  \item \textbf{Duration}
    \begin{itemize}
      \item Rappresenta il periodo di tempo trascorso dall'utente nell'ambiente di realtà virtuale.
    \end{itemize}
  \item \textbf{Motion sickness}
    \begin{itemize}
      \item Questa variabile indica la valutazione auto-riferita dell'utente della cinetosi sperimentata durante l’esperienza di utilizzo del visiore VR
    \end{itemize}
  \item \textbf{Immersion Level}
    \begin{itemize}
      \item Questa variabile misura il livello di immersione dell'utente durante l'esperienza VR. Rappresenta il livello soggettivo di immersione riportato dall'utente e può essere misurato su una scala da 1 a 5, dove 5 indica il livello di immersione più alto.
    \end{itemize}
    \newline
        \newline
    Nel dataset, non tutte le caratteristiche presenti risultano essere altrettanto rilevanti per il nostro agente intelligente. Di conseguenza, abbiamo selezionato le caratteristiche più significative per le analisi del nostro agente intelligente, focalizzandoci specificamente sull'elaborazione dei dati necessari al nostro modello di Machine Learning. 
    \newline
    Tra le caratteristiche  \textbf{considerate}, vi sono le seguenti sei:
    \item \textbf{User ID}
    \item \textbf{Age}
    \item \textbf{Gender}
    \item \textbf{VR Headset Type}
    \item \textbf{Duration}
    \item \textbf{Motion sickness}
    \item \textbf{Immersion Level}
    \end{itemize}
    La durata (Duration) si configura come variabile dipendente, la quale il nostro agente intelligente dovrà stimare.
}