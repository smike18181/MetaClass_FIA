\subsection{Incremento del dataset}
\fancyhead{}    % reset header
\fancyhead[R]{Incremento del dataset}
\label{paragrafo 2.3}
\par{
Il \textbf{retraining}, o riaddestramento, rappresenta una fase \textbf{cruciale} nel ciclo di vita di un modulo di intelligenza artificiale, si riferisce al \textbf{processo di aggiornamento e miglioramento} del modello di intelligenza al fine di \textbf{migliorare la qualità delle predizioni} effettuate.
\newline
La \textbf{motivazione} principale che ci spinge a voler effettuare il retraining del nostro modulo di intelligenza artificiale risiede nella \textbf{scarsità dei dati} all'interno del dataset utilizzato per il training iniziale del modello; questa carenza di dati ha un \textbf{impatto negativo sulla qualità delle previsioni} del modello, di conseguenza, è essenziale aggiornare il modello con nuovi dati per migliorarne l'accuratezza e l'efficacia.
\newline
\newline
Per quanto riguarda le informazioni relative alla \textbf{motion sickness e all'immersion} level, la strategia di acquisizione dei nuovi dati si basa sulla raccolta dei report inseriti da ogni utente all'interno del sistema; 
\textbf{La durata del meeting} per singolo utente, invece, viene rilevata direttamente dall'applicativo VR, inoltre, le informazioni relative al \textbf{sesso ed età} di ciascun utente vengono acquisite automaticamente dal sistema stesso.
\newline
\newline
Questi dati vengono raccolti al termine di ogni meeting e costituiscono la base per la predizione della durata consigliata per il meeting.
\newline
\newline
Una volta raccolte le \textbf{nuove istanze} di dati, il processo di retraining prevede di \textbf{addestrare nuovamente} il modello utilizzando queste nuove informazioni. Per facilitare questo processo, abbiamo sviluppato un \textbf{modulo Python dedicato}, come descritto nel \ref{paragrafo 5.3} \newline
Questo modulo gestisce l'aggiornamento del modello, l'adattamento dei parametri e la valutazione delle prestazioni del modello aggiornato.
\newline
}