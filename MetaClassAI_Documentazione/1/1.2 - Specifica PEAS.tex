\subsection{Specifica PEAS}
\fancyhead{}    % reset header
\fancyhead[R]{Specifica PEAS}
\label{paragrafo 1.2}
\par{
\begin{itemize}
    \item \textbf{Performance}: La misura di performance dell'agente è la precisione con cui si avvicina alla reale durata media del meeting quando questa viene proposta in fase di programmazione;
    \item \textbf{Environment}: L'ambiente in cui opera l'agente è l'insieme dei feedback degli utenti a fine meeting del nostro applicativo;
    \item \textbf{Actuators}:Gli attuatori a disposizione dell'agente sono le caselle di input del range orario, che presenta già l'offset basato sulla stima fatta sulla fascia oraria proposta quando si va a schedulare un meeting sul sito \url{https://metaclass.commigo.it};
    \item \textbf{Sensors}: I sensori tramite il quale l'agente recepisce stimoli dall'ambiente sono i report compilabili sul sito \url{https://metaclass.commigo.it} e sul tempo di utilizzo del visore per singolo utente all'interno di un meeting;
\end{itemize}
}
