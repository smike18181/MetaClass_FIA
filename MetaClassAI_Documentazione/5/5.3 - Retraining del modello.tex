\subsection{Retraining del modello}
\fancyhead{}    % reset header
\fancyhead[R]{Retraining del modello}

Una volta inserite le nuove istanze al termine della compilazione del questionario, il modello deve essere riallenato considerando i nuovi valori inseriti.
Per fare ciò è stato necessario creare un modulo python che contenesse un interprete e un'ambiente virtuale in grado di installare i moduli e le liberie necessarie per l'esecuzione di un file python (trainingModel.py) contenente il codice necessario per fare ciò.

\begin{bashCode}
    \begin{tcolorbox}[colback=bgTitleRed, breakable, fontsize=\footnotesize]
    # coding: utf-8
    import pandas as pd
    
    FILE_NAME = "../src/main/resources/ModuloAI/data.csv"
    
    # Verifica il contenuto del file
    try:
        df = pd.read_csv(FILE_NAME)
    except pd.errors.EmptyDataError:
        print("Il file CSV è vuoto.")
    except pd.errors.ParserError:
        print("Errore di parsing del file CSV.")
    except Exception as e:
        print(e)
    \end{tcolorbox}
    \caption{Prelievo del dataset}
    \label{lst:python_code}
\end{bashCode}


In questa parte si va a prelevare il dataset presente in una cartella specifica del nostro progetto.
Tale dataset è stato estratto in formato csv e tramite il modulo pandas si andrà ad ottenere il DataFrame contente tutte le tuple del dataset.
In caso di dataset vuoto viene ritornata un'eccezione (gestita dal metodo chiamante java) e stessa cosa succede per errori di parsing del file csv e per errori generici.
%immagine feature preparation
in quest'altra figura si applica