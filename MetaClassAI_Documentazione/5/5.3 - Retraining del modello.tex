\subsection{Retraining del modello}
\fancyhead{}    % reset header
\fancyhead[R]{Retraining del modello}

Una volta inserite le nuove istanze al termine della compilazione del questionario, il modello deve essere \textbf{riallenato} considerando i nuovi valori inseriti.\vspace{1ex}
Per fare ciò è stato necessario creare un \textbf{modulo python} che contenesse un \textbf{interprete} e \textbf{un'ambiente virtuale} in grado di installare tutte le liberie necessarie per l'esecuzione di un file python \textcolor{gray!50!black}{\textit{(trainingModel.py)}} contenente il codice necessario per fare ciò. \\ 

\begin{lstlisting}[caption=Prelievo del dataset]
    # coding: utf-8
    import pandas as pd
    
    FILE_NAME = "../src/main/resources/ModuloAI/data.csv"
    
    # Verifica il contenuto del file
    try:
        df = pd.read_csv(FILE_NAME)
    except pd.errors.EmptyDataError:
        print("Il file CSV e' vuoto.")
    except pd.errors.ParserError:
        print("Errore di parsing del file CSV.")
    except Exception as e:
        print(e)
\end{lstlisting} 

In questa parte si va a prelevare il dataset presente in una cartella specifica del nostro progetto.
Tale dataset è stato estratto in formato \textbf{csv} e tramite il modulo \textbf{pandas} si andrà ad ottenere il \textbf{DataFrame} contente tutte le tuple del dataset.\vspace{1ex}\\
In caso di dataset vuoto viene ritornata un'eccezione (gestita dal metodo chiamante java) come succede anche per errori di \textbf{parsing} del file csv e per errori generici. \\

\begin{lstlisting}[caption=Feature preparation]
   from sklearn.ensemble import RandomForestRegressor
   from sklearn.preprocessing import RobustScaler
   from sklearn2pmml import sklearn2pmml
   from sklearn2pmml.pipeline import PMMLPipeline

   #Scelta variabile dipendente (y) e indipendenti (X)
   X=df.drop(columns=['Duration','VRHeadset',"UserID"])   #feature selection
   y=df.Duration

   # Adattamento e trasformazione dei dati con RobustScaler
   scaler = RobustScaler()
   X_scaled = scaler.fit_transform(X)

  # Creazione del modello di regressione con Random Forest
  model = RandomForestRegressor()
  model.fit(X_scaled, y)
\end{lstlisting} 

