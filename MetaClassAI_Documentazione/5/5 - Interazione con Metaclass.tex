\section{ Inteazione con MetaClass }
\fancyhead[]{}
\fancyhead[R]{ Interazione con MetaClass }
MetaClass è una Companion App che supporta funzionalità critiche di MetaClassVR, un'applicazione a realtà aumentata per la creazione e di meeting e stanze virtuali. \\
Prima della creazione di MetaClass, tutte le funzionalità venivano gestite con l'ausilio di un visore che per novizi del mondo della realtà virtuale, non era il massimo in termini di usabilità.
MetaClass quindi viene incontro a queste esigenze, mostrando un'interfaccia chiara e funzionale su tutti i requisiti funzionali dell'applicativo inizale, quali:
\par{
\begin{itemize}
    \item \textbf{creazione e modifica stanze}
    \item \textbf{scheduling meetings}
    \item \textbf{modifica dati personali}
    \item \textbf{...}
\end{itemize}
}
MetaClass aggiunge però una nuova funzionalità, ovvero la possibilità di far compilare all'utente un questionario (descritto nel paragrafo \ref{paragrafo 2.3}). 
Il tutto al fine di prendere tali valutazioni e aggiungere nuove istanze al nostro dataset per cercare di migliorare i problemi di underfitting dovuti alla scarsità dei dati del nostro modello. (come è stato spiegato nel paragrafo \ref{paragrafo 2.1}). \\

\newpage
\subsection{Stima durata dei meeting}
\fancyhead{}    % reset header
\fancyhead[R]{Stima durata dei meeting}
\label{paragrafo 5.1}

L'essenza del nostro regressore, come è stato spiegato nel paragrafo \ref{paragrafo 1.3}, è quella di rilevare i dati di un'utente per poi stimare la durata ideale di un meeting che tale utente preferisce seguire.
La stima però deve essere eseguita su tutti gli utenti presenti in una stanza per far si che l'organizzatore di una stanza, possa essere a conoscenza della durata ideale media da effettuare in una determinata stanza.
Il modello quindi deve effettuare stime su ogni partecipante della stanza per poi ritornare una durata media dei meetings. Qui sotto riportato un'estratto di codice di metaclass che consente di fare ciò: \\
\begin{lstlisting}[caption=Stima durata meeting ideale]
  @Override
  public Double getDurataMeeting(Long idStanza) throws RuntimeException403, ServerRuntimeException {
    // controllo la stanza se esiste
    Stanza s;
    if ((s = stanzaRepository.findStanzaById(idStanza)) == null) {
      throw new RuntimeException403("non è possibile effettuare la stima: stanza non trovata");
    }
    // mi recapito la lista di utenti in stanza
    List<Utente> users;
    if ((users = statoPartecipazioneRepository.findUtentiInStanza(s.getId())) == null) {
      throw new ServerRuntimeException(
          "non è possibile effettuare la stima: " + "errore nella ricerca dei partecipanti");
    }
    double mediaDuration = 0;

    for (Utente u : users) {
      mediaDuration += requestPrediction(u);
    }
    return mediaDuration / users.size();
  }
\end{lstlisting}
Dove vengono estratti tutti gli utenti presenti in stanza per poi invocare il metodo \textit{requestPrediction()}
che prende in input un utente ed effettua la stima su di esso.

Il tutto viene fatto per tutti gli utenti e le stime eseguite vengono sommate nella variabile \textit{mediaDuration} per poi ritornare la media di tali stime. \\
Il metodo requestPrediction() ha il compito di preparare i dati dell'utente per renderli compatibili con il regressore (memorizzato in un file PMML) per poi effettuare la stima. 


\href{https://github.com/Everysimo/MetaClass}{\textcolor{blue}{https://github.com/Everysimo/MetaClass}} per ulteriori dettagli sul codice relativo alle stime.
\newpage
\subsection{Aggiunta istanze}
\fancyhead{}    % reset header
\fancyhead[R]{Aggiunta istanze}
\newpage
\subsection{Retraining del modello}
\fancyhead{}    % reset header
\fancyhead[R]{Retraining del modello}

Una volta inserite le nuove istanze al termine della compilazione del questionario, il modello deve essere \textbf{riallenato} considerando i nuovi valori inseriti.\vspace{1ex}
Per fare ciò è stato necessario creare un \textbf{modulo python} che contenesse un \textbf{interprete} e \textbf{un'ambiente virtuale} in grado di installare tutte le liberie necessarie per l'esecuzione di un file python \textcolor{gray!50!black}{\textit{(trainingModel.py)}} contenente il codice necessario per fare ciò. \\ 

\begin{lstlisting}[caption=Prelievo del dataset]
    # coding: utf-8
    import pandas as pd
    
    FILE_NAME = "../src/main/resources/ModuloAI/data.csv"
    
    # Verifica il contenuto del file
    try:
        df = pd.read_csv(FILE_NAME)
    except pd.errors.EmptyDataError:
        print("Il file CSV e' vuoto.")
    except pd.errors.ParserError:
        print("Errore di parsing del file CSV.")
    except Exception as e:
        print(e)
\end{lstlisting} 

In questa parte si va a prelevare il dataset presente in una cartella specifica del nostro progetto.
Tale dataset è stato estratto in formato \textbf{csv} e tramite il modulo \textbf{pandas} si andrà ad ottenere il \textbf{DataFrame} contente tutte le tuple del dataset.\vspace{1ex}\\
In caso di dataset vuoto viene ritornata un'eccezione (gestita dal metodo chiamante java) come succede anche per errori di \textbf{parsing} del file csv e per errori generici. \\

\begin{lstlisting}[caption=Feature preparation]
   from sklearn.ensemble import RandomForestRegressor
   from sklearn.preprocessing import RobustScaler
   from sklearn2pmml import sklearn2pmml
   from sklearn2pmml.pipeline import PMMLPipeline

   #Scelta variabile dipendente (y) e indipendenti (X)
   X=df.drop(columns=['Duration','VRHeadset',"UserID"])   #feature selection
   y=df.Duration

   # Adattamento e trasformazione dei dati con RobustScaler
   scaler = RobustScaler()
   X_scaled = scaler.fit_transform(X)

  # Creazione del modello di regressione con Random Forest
  model = RandomForestRegressor()
  model.fit(X_scaled, y)
\end{lstlisting} 

In quest altro blocco di codice si va a trasformare il dataset e i dati al suo interno per prepararli al \textbf{training} del modello. ( pragrafo \ref{paragrafo 3.1} per una descrizione più dettagliata). \vspace{1ex}\\ 
Dopo aver applicato il \textbf{metodo empirico} su vari algoritmi e per ogni algortimo aver applicato \textbf{3 tipologie di scaling} diverse, si è visto che l'algoritmo che si adatta meglio ai dati è il \textbf{RandomForestRegressor} applicando però la \textbf{RobustScaler} sui dati.
Per preparare i dati vi sono una serie di passi da seguire:
\par{
\begin{itemize}
   \item \textbf{feature selection}: vengono selezionate le feature \textbf{caratterizzanti} del problema in esame: \textit{Age, Gender, MotionSickness, ImmersionLevel}. Feature come \textit{VRHeadset} non sono state considerate in quanto i visori che verranno utilizzati nelle lezioni virtuali hanno headset diversi dalle 3 tipologie presenti nel dataset. In più, il regressore dovrà stimare la durata ideale di un meeting in una determinata stanza e per fare ciò si basa sul \textit{MotionSickess} che racchiude tutti i fastidi provati durante meeting precedenti, incluso anche quello dell'headset. 
   \item \textbf{feature scaling}: come detto prima, il \textbf{RobustScaler} consente di fare stime più accurate rispetto altri scaler. Da come si legge nel codice, viene applicato lo scaling su X che corrisponde all'insieme di tuple presenti nel dataset su cui è stata applicata la feature selection
   \item \textbf{feature balancing e feature cleaning}: non 
   sono stati applicati per i motivi descritti nel \textit{paragrafo \ref{paragrafo 3.1}}.
\end{itemize}
} \\ 

\begin{lstlisting}[caption=Conversione in un file PMML]
   # Creazione del PMMLPipeline con il modello già addestrato
   pipeline = PMMLPipeline([("Regression", model)])

   # Tentativo di estrazione del pipeline in un file PMML con gestione delle eccezioni
   try:
      sklearn2pmml(pipeline, "RegressoreDurataMeeting.pmml", with_repr=True)
   except Exception as e:
      print("Si è verificato un errore durante l'estrazione del pipeline in un file PMML:")
      print(e)
\end{lstlisting} 

In quest'ultima parte si andrà a convertire il modello già addestrato in un \textbf{file PMML} che verrà utilizzato per compiere stime in un altro modulo di MetaClass (descritto nel paragrafo \ref{paragrafo 5.1}). \vspace{1ex} \\
Per poter compiere la conversione è necessario creare una \textbf{pipeline} contenente il \textbf{modello addestrato}. l'istanza model inoltre conterrà anche tutti i dati del dataset con relativa separazione tra variabili dipendenti e indipendenti. \\
Infine in blocco try/except si creare il file che verrà salvato nella directory corrente del modulo python appena descritto.






