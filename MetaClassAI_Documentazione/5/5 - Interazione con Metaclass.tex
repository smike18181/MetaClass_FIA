\section{ Inteazione con MetaClass }
\fancyhead[]{}
\fancyhead[R]{ Interazione con MetaClass }
MetaClass è una Companion App che supporta funzionalità critiche di MetaClassVR, un'applicazione a realtà aumentata per la creazione e di meeting e stanze virtuali. \\
Prima della creazione di MetaClass, tutte le funzionalità venivano gestite con l'ausilio di un visore che per novizi del mondo della realtà virtuale, non era il massimo in termini di usabilità.
MetaClass quindi viene incontro a queste esigenze, mostrando un'interfaccia chiara e funzionale su tutti i requisiti funzionali dell'applicativo inizale, quali:
\par{
\begin{itemize}
    \item \textbf{creazione e modifica stanze}
    \item \textbf{scheduling meetings}
    \item \textbf{modifica dati personali}
    \item \textbf{...}
\end{itemize}
}
MetaClass aggiunge però una nuova funzionalità, ovvero la possibilità di far compilare all'utente un questionario (descritto nel paragrafo \ref{paragrafo 2.3}). 
Il tutto al fine di prendere tali valutazioni e aggiungere nuove istanze al nostro dataset per cercare di migliorare i problemi di underfitting dovuti alla scarsità dei dati del nostro modello. (come è stato spiegato nel paragrafo \ref{paragrafo 2.1}). \\

\newpage
\subsection{Stima durata dei meeting}
\fancyhead{}    % reset header
\fancyhead[R]{Stima durata dei meeting}
\label{paragrafo 5.1}
\newpage
\subsection{Aggiunta istanze}
\fancyhead{}    % reset header
\fancyhead[R]{Aggiunta istanze}
\label{paragrafo 5.2}

MetaClass,come descritto prima, consente la compilazione di un questionario, dopo un meeting a cui ha partecipato l'utente, è terminato.
I dati del questionario vengono presi per incrementare il dataset, infatti oltre al motionSickenss e al Immersivity Level, si prelevano dati come età e sesso.
Per poter inserire la tupla nel dataset ci occorre però un modo per poter ricavare lo \textit{UserID} dell'ultima istanza del dataset per incrementarla e farla diventare lo \textit{UserID} della prossima tupla da inserire.
Per fare ciò è stato necessario utilizzare la libreria \textit{org.apache.commons} per poter fare la lettura e la scrittura in un file formato csv (formato di estrazione del dataset). \\
\begin{lstlisting}
import org.apache.commons.csv.CSVFormat;
import org.apache.commons.csv.CSVParser;
import org.apache.commons.csv.CSVPrinter;
import org.apache.commons.csv.CSVRecord;
\end{lstlisting}
Questi sono gli oggetti utilizzati per accedere a un file csv, in particolare:
\par{
\begin{itemize}
    \item \textbf{CSVFormat}:
     Questo oggetto definisce il formato del file CSV, specificando opzioni come il separatore di colonne, il carattere di escape e altri parametri di formattazione. Viene utilizzato per configurare la lettura e la scrittura dei file CSV.
     \item \textbf{CSVParser}:
     Questo oggetto è responsabile della lettura di un file CSV, analizzandolo e convertendo i dati in oggetti di tipo CSVRecord. Utilizza il CSVFormat per interpretare correttamente la struttura del file CSV.
     \item \textbf{CSVPrinter}:
      Questo oggetto viene utilizzato per la scrittura di dati in un file CSV. Accetta dati in forma di oggetti CSVRecord e li formatta secondo il CSVFormat specificato, scrivendoli nel file CSV di output.
      \item \textbf{CSVRecord}:
      Rappresenta una singola riga di dati in un file CSV. I valori dei singoli campi sono accessibili attraverso metodi come get o tramite l'indicizzazione. Viene utilizzato sia durante la lettura dei file CSV da CSVParser che durante la scrittura da CSVPrinter.
\end{itemize}
}

Ottenuta la tupla risultante, è possibile inserirla nel dataset: \\

\begin{lstlisting}
    // Aggiunta della nuova tupla di valori al CSV
        csvPrinter.printRecord(
            ultimoUserId + 1, // UserID
            periodo.getYears(), // Age
                (u.getSesso().equals("M") ? 2 : (u.getSesso().equals("F")  ? 1 : 0)), // Gender
            null, // VR Headset (non utilizzato e rimosso nella feature selection)
            (double) durata.toMinutes(), // Duration
            immersionLevel, // ImmersionLevel
            motionSickness); // MotionSickness
\end{lstlisting}

Da come si può notare il sesso è stato espresso come un valore numerico (per consentire lo scaling durante il training del modello) e la cella relativa all'headSet è nulla in quanto verrà eliminata a prescindere durante le feature selection. \\
Infine una questione importante ce l'ha l'attributo Duration che è stato ricavato andando a fare una media di tutti i minuti in cui è stato l'utente all'interno dei meetings (considerando però anche le altre stanze in cui ha accesso).  

Infine dopo l'inserimento di dati era necessario stabilire quando effetturare il \textbf{retraining del modello} avendo a disposizione 2 alternative:
\par{
\begin{itemize}
   \item \textbf{allenarlo dopo ogni utente aggiunto}
   \item \textbf{allenarlo dopo n inserimenti}
\end{itemize}
}
A primo impatto sembra che la prima opzione sia la migliore, in quanto si ha fin da subito il modello riallenato e quindi fornire stime fin da subito più accurate, ma questo porta come svantaggio un esecuzione lenta nella compilazione del questionario. In parole povere, quando l'utente compila un questionario, dovrà attendere diversi secondi per avere un feedback di avvenuta compilazione, rendendo quindi il sistema meno usabile.
A valle di ciò si è deciso di implentare la seconda opzione, dove \textbf{ad ogni 100 inserimenti} veniva inserita effettuato il retraining:
\begin{lstlisting}
     //incremento il contatore che mi tiene traccia di quante tuple vengono aggiunte
        counter++;
      //ogni 100 inserimenti avviene il retraining
      if ((counter% 100) == 0) {
        // ri-training del modello
        trainingModel();
      }
\end{lstlisting}
Per fare ciò è servito un contatore statico che veniva incrementato ad ogni inserimento, poi in un if veniva controllato se il contatore ha un modulo pari a 100 (cioè se sono stati inserite 100 tuple).


\newpage
\subsection{Retraining del modello}
\fancyhead{}    % reset header
\fancyhead[R]{Retraining del modello}

Una volta inserite le nuove istanze al termine della compilazione del questionario, il modello deve essere \textbf{riallenato} considerando i nuovi valori inseriti.\vspace{1ex}
Per fare ciò è stato necessario creare un \textbf{modulo python} che contenesse un \textbf{interprete} e \textbf{un'ambiente virtuale} in grado di installare tutte le liberie necessarie per l'esecuzione di un file python \textcolor{gray!50!black}{\textit{(trainingModel.py)}} contenente il codice necessario per fare ciò. \\ 

\begin{lstlisting}[caption=Prelievo del dataset]
    # coding: utf-8
    import pandas as pd
    
    FILE_NAME = "../src/main/resources/ModuloAI/data.csv"
    
    # Verifica il contenuto del file
    try:
        df = pd.read_csv(FILE_NAME)
    except pd.errors.EmptyDataError:
        print("Il file CSV e' vuoto.")
    except pd.errors.ParserError:
        print("Errore di parsing del file CSV.")
    except Exception as e:
        print(e)
\end{lstlisting} 

In questa parte si va a prelevare il dataset presente in una cartella specifica del nostro progetto.
Tale dataset è stato estratto in formato \textbf{csv} e tramite il modulo \textbf{pandas} si andrà ad ottenere il \textbf{DataFrame} contente tutte le tuple del dataset.\vspace{1ex}\\
In caso di dataset vuoto viene ritornata un'eccezione (gestita dal metodo chiamante java) come succede anche per errori di \textbf{parsing} del file csv e per errori generici. \\

\begin{lstlisting}[caption=Feature preparation]
   from sklearn.ensemble import RandomForestRegressor
   from sklearn.preprocessing import RobustScaler
   from sklearn2pmml import sklearn2pmml
   from sklearn2pmml.pipeline import PMMLPipeline

   #Scelta variabile dipendente (y) e indipendenti (X)
   X=df.drop(columns=['Duration','VRHeadset',"UserID"])   #feature selection
   y=df.Duration

   # Adattamento e trasformazione dei dati con RobustScaler
   scaler = RobustScaler()
   X_scaled = scaler.fit_transform(X)

  # Creazione del modello di regressione con Random Forest
  model = RandomForestRegressor()
  model.fit(X_scaled, y)
\end{lstlisting} 

In quest altro blocco di codice si va a trasformare il dataset e i dati al suo interno per prepararli al \textbf{training} del modello. ( pragrafo \ref{paragrafo 3.1} per una descrizione più dettagliata). \vspace{1ex}\\ 
Dopo aver applicato il \textbf{metodo empirico} su vari algoritmi e per ogni algortimo aver applicato \textbf{3 tipologie di scaling} diverse, si è visto che l'algoritmo che si adatta meglio ai dati è il \textbf{RandomForestRegressor} applicando però la \textbf{RobustScaler} sui dati.
Per preparare i dati vi sono una serie di passi da seguire:
\par{
\begin{itemize}
   \item \textbf{feature selection}: vengono selezionate le feature \textbf{caratterizzanti} del problema in esame: \textit{Age, Gender, MotionSickness, ImmersionLevel}. Feature come \textit{VRHeadset} non sono state considerate in quanto i visori che verranno utilizzati nelle lezioni virtuali hanno headset diversi dalle 3 tipologie presenti nel dataset. In più, il regressore dovrà stimare la durata ideale di un meeting in una determinata stanza e per fare ciò si basa sul \textit{MotionSickess} che racchiude tutti i fastidi provati durante meeting precedenti, incluso anche quello dell'headset. 
   \item \textbf{feature scaling}: come detto prima, il \textbf{RobustScaler} consente di fare stime più accurate rispetto altri scaler. Da come si legge nel codice, viene applicato lo scaling su X che corrisponde all'insieme di tuple presenti nel dataset su cui è stata applicata la feature selection
   \item \textbf{feature balancing e feature cleaning}: non 
   sono stati applicati per i motivi descritti nel \textit{paragrafo \ref{paragrafo 3.1}}.
\end{itemize}
} \\ 

\begin{lstlisting}[caption=Conversione in un file PMML]
   # Creazione del PMMLPipeline con il modello già addestrato
   pipeline = PMMLPipeline([("Regression", model)])

   # Tentativo di estrazione del pipeline in un file PMML con gestione delle eccezioni
   try:
      sklearn2pmml(pipeline, "RegressoreDurataMeeting.pmml", with_repr=True)
   except Exception as e:
      print("Si è verificato un errore durante l'estrazione del pipeline in un file PMML:")
      print(e)
\end{lstlisting} 

In quest'ultima parte si andrà a convertire il modello già addestrato in un \textbf{file PMML} che verrà utilizzato per compiere stime in un altro modulo di MetaClass (descritto nel paragrafo \ref{paragrafo 5.1}). \vspace{1ex} \\
Per poter compiere la conversione è necessario creare una \textbf{pipeline} contenente il \textbf{modello addestrato}. l'istanza model inoltre conterrà anche tutti i dati del dataset con relativa separazione tra variabili dipendenti e indipendenti. \\
Infine in blocco try/except si creare il file che verrà salvato nella directory corrente del modulo python appena descritto.






