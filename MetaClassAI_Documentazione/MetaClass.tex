\documentclass[a4paper, twoside, openright]{article}
\usepackage[T1]{fontenc}
\usepackage[utf8]{inputenc}
\usepackage[english,italian]{babel}
\usepackage{graphicx}
\usepackage{lipsum}
\usepackage[a4paper,top=3cm,bottom=3cm,left=3cm,right=3cm]{geometry}
\usepackage[fontsize=12pt]{scrextend}
\raggedbottom
\linespread{1.25}
\usepackage{fancyhdr}
\usepackage{amsmath}

\usepackage{xcolor}
\usepackage{listings}
\usepackage{caption}

% Definisci i colori che desideri utilizzare
\definecolor{codebackground}{rgb}{0.95,0.95,0.95}
\definecolor{codeframe}{rgb}{0.75,0.75,0.75}
\definecolor{codekeyword}{rgb}{0.37,0.08,0.25}

% Impostazioni per il blocco di codice
\lstset{
    language=Python,
    basicstyle=\ttfamily\small,
    keywordstyle=\color{codekeyword}\bfseries,
    backgroundcolor=\color{codebackground},
    frame=single,
    rulecolor=\color{codeframe},
    breaklines=true,
    showstringspaces=false,
    captionpos=b,
}


\pagestyle{fancy}
\fancypagestyle{plain}{%
  \renewcommand{\headrulewidth}{15pt}%
  \fancyhf{}%
}

\usepackage{subfig}
\usepackage{float}
\graphicspath{ {./images/} }

\definecolor{linkColor}{RGB}{2,11,120}
\definecolor{printLinkColor}{RGB}{0,0,0}

\usepackage[colorlinks=true, allcolors=printLinkColor]{hyperref}
\newcommand\anchor[2]{%
  \href{#2}{#1}\footnote{\url{#2}}%
}

\usepackage{tgschola}

\definecolor{bgTitleRed}{RGB}{85,45,50}
\fancyhf{}
\fancyfoot[R]{\thepage}


\begin{document}

%caption dei listings è in bold
\captionsetup[lstlisting]{labelfont={bf}}

\begin{figure}[H]
    \centering
    \includegraphics[width=7cm]{img/logo.png}
\end{figure}

\begin{center}
    \LARGE{UNIVERSITÀ DI SALERNO}
    \vspace{1mm}
    \\ \large{DIPARTIMENTO DI INFORMATICA }
    \vspace{5mm}
\end{center}

% Titolo e
\vspace{10mm}
\begin{center}
    {\LARGE{\bf \textit{MetaClassAI}\\\vspace{5mm} Regressore per la stima della durata ideale dei meeting}}
\end{center}

\vspace{5mm}
\begin{center}
    \centering
    \large{Repository GitHub:}
    \href{https://github.com/smike18181/MetaClass_FIA}{\textcolor{blue}{https://github.com/smike18181/MetaClass\_FIA}}
\end{center}

\vspace{20mm}

% Definzione autori
\begin{center}
        {\centering
        \large {Compilato da:} \\
        \vspace{2mm}
        \normalsize \textbf{Pesce Michele} \\
        \vspace{2mm}
        \normalsize \textbf{Alberti Salvatore} \\
        \vspace{2mm}
        \normalsize \textbf{Gatto Francesco} \\
        \vspace{2mm}
        \normalsize \textbf{Cavaliere Domenico} \\}   
\end{center}    

% Index page
\newpage
\fancyhead[R]{ Indice } %RO=right odd, LE=left even
\tableofcontents

% Footer setting
\newpage
\fancyfoot[R]{\thepage}

% Chapters
% 1 - Definizione del problema.tex
\section{Definizione del Problema}

\fancyhead{}    % reset header
\fancyhead[R]{Definizione del problema}

\subsection{Obiettivi}
\fancyhead{}    % reset header
\fancyhead[R]{Obiettivi}
\par{
Obiettivo del progetto è quello di xyz
}
\subsection{Specifica PEAS}
\fancyhead{}    % reset header
\fancyhead[R]{Specifica PEAS}
\par{
\begin{itemize}
    \item \textbf{Performance}: La misura di performance dell'agente è la precisione con cui si avvicina alla reale durata media del meeting quando questa viene proposta in fase di programmazione;
    \item \textbf{Environment}: L'ambiente in cui opera l'agente è l'insieme dei feedback degli utenti a fine meeting del nostro applicativo;
    \item \textbf{Actuators}:Gli attuatori a disposizione dell'agente sono le caselle di input del range orario, che presenta già l'offset basato sulla stima fatta sulla fascia oraria proposta quando si va a schedulare un meeting sul sito \url{https://metaclass.commigo.it};
    \item \textbf{Sensors}: I sensori tramite il quale l'agente recepisce stimoli dall'ambiente sono i report compilabili sul sito \url{https://metaclass.commigo.it} e sul tempo di utilizzo del visore per singolo utente all'interno di un meeting;
\end{itemize}
}

\subsubsection{Caratteristche dell'ambiente}
\fancyhead{}    % reset header
\fancyhead[R]{Caratteristche dell'ambiente}
\subsection{Analisi del problema}
\fancyhead{}    % reset header
\fancyhead[R]{Analisi del problema}
% Altri contenuti, se presenti



\newpage
\section{Dataset}
\fancyhead[]{}
\fancyhead[R]{ Dataset }

\subsection{Scelta del dataset}
\fancyhead{}    % reset header
\fancyhead[R]{Scelta del dataset}
\subsection{Analisi Dataset e ei dati al suo interno}
\fancyhead{}    % reset header
\fancyhead[R]{Analisi Dataset e ei dati al suo interno}
\par{
Le fonti del dataset sono costituite principalmente da informazioni ricavate da studi effettuati in laboratorio.
\textbf{il Dataset si compone delle seguenti sette caratteristiche:}\newline
\begin{itemize}
  \item Primo elemento
  \item Secondo elemento
  \item Terzo elemento
\end{itemize}
}
\subsection{Incremento del dataset}
\fancyhead{}    % reset header
\fancyhead[R]{Incremento del dataset}

\newpage
\section{ Ingegnerizzazione dei dati }
\fancyhead[]{}
\fancyhead[R]{ Ingegnerizzazione dei dati }

\subsection{Data cleaning}
\fancyhead{}    % reset header
\fancyhead[R]{Data cleaning}
Nella prima fase di ingegnerizzazione dei dati ci siamo soffermati sul data cleaning, ovvero siamo andati ad analizzare i dati valutando se ci fossero valori nulli o non validi.
Da un analisi preliminare dei dati a nostra disposizione è risultato che:

\begin{figure}[h]
    \centering
    \includegraphics[width=0.75\textwidth]{MetaClassAI_Documentazione/3/img/ValoriNULLDataset.png}
    \caption{Esempio di verifica dei dati per valori nulli}
    \label{fig:verifica-null-dataset}
\end{figure}

\begin{figure}[h]
    \centering
    \includegraphics[width=0.75\textwidth]{MetaClassAI_Documentazione/3/img/ValoriNADataset.png}
    \caption{Esempio di verifica dei dati per valori N/A}
    \label{fig:verifica-NA-dataset}
\end{figure}
Come si può notare, non sono presenti dati nulli o invalidi tra quelli presenti, per cui non sono state necessarie tecniche di Data cleaning.

\subsection{Feature scaling}
\fancyhead{}    % reset header
\fancyhead[R]{Feature scaling}
\subsection{Feature selection}
\fancyhead{}    % reset header
\fancyhead[R]{Feature selection}
\subsection{Data balancing}
\fancyhead{}    % reset header
\fancyhead[R]{Data balancing}

\newpage
\section{ \textbf{Soluzione}\textbf{ al problema} }
\fancyhead[]{}
\fancyhead[R]{Soluzione al problema}


\subsection{Model evaluation}
\fancyhead{}    % reset header
\fancyhead[R]{Model evaluation}
\subsection{Regressione}
\fancyhead{}    % reset header
\fancyhead[R]{Regressione}
\subsubsection{Regressione lineare}
\fancyhead{}    % reset header
\fancyhead[R]{Regressione lineare}
v\subsection{Algoritmi}
\fancyhead{}    % reset header
\fancyhead[R]{Algoritmi}


\subsubsection{Algoritmo scelto}
\fancyhead{}    % reset header
\fancyhead[R]{Algoritmo scelto}
\input{MetaClassAI_Documentazione/4/4.3.1 - Creazione modello finale}



\newpage
\section{ Inteazione con MetaClass }
\fancyhead[]{}
\fancyhead[R]{ Interazione con MetaClass }

\subsection{Stima durata dei meeting}
\fancyhead{}    % reset header
\fancyhead[R]{Stima durata dei meeting}
\label{paragrafo 5.1}

L'essenza del nostro regressore, come è stato spiegato nel paragrafo \ref{paragrafo 1.3}, è quella di rilevare i dati di un'utente per poi stimare la durata ideale di un meeting che tale utente preferisce seguire.
La stima però deve essere eseguita su tutti gli utenti presenti in una stanza per far si che l'organizzatore di una stanza, possa essere a conoscenza della durata ideale media da effettuare in una determinata stanza.
Il modello quindi deve effettuare stime su ogni partecipante della stanza per poi ritornare una durata media dei meetings. Qui sotto riportato un'estratto di codice di metaclass che consente di fare ciò: \\
\begin{lstlisting}[caption=Stima durata meeting ideale]
  @Override
  public Double getDurataMeeting(Long idStanza) throws RuntimeException403, ServerRuntimeException {
    // controllo la stanza se esiste
    Stanza s;
    if ((s = stanzaRepository.findStanzaById(idStanza)) == null) {
      throw new RuntimeException403("non è possibile effettuare la stima: stanza non trovata");
    }
    // mi recapito la lista di utenti in stanza
    List<Utente> users;
    if ((users = statoPartecipazioneRepository.findUtentiInStanza(s.getId())) == null) {
      throw new ServerRuntimeException(
          "non è possibile effettuare la stima: " + "errore nella ricerca dei partecipanti");
    }
    double mediaDuration = 0;

    for (Utente u : users) {
      mediaDuration += requestPrediction(u);
    }
    return mediaDuration / users.size();
  }
\end{lstlisting}
Dove vengono estratti tutti gli utenti presenti in stanza per poi invocare il metodo \textit{requestPrediction()}
che prende in input un utente ed effettua la stima su di esso.

Il tutto viene fatto per tutti gli utenti e le stime eseguite vengono sommate nella variabile \textit{mediaDuration} per poi ritornare la media di tali stime. \\
Il metodo requestPrediction() ha il compito di preparare i dati dell'utente per renderli compatibili con il regressore (memorizzato in un file PMML) per poi effettuare la stima. 


\href{https://github.com/Everysimo/MetaClass}{\textcolor{blue}{https://github.com/Everysimo/MetaClass}} per ulteriori dettagli sul codice relativo alle stime.
\subsection{Aggiunta istanze}
\fancyhead{}    % reset header
\fancyhead[R]{Aggiunta istanze}
\subsection{Retraining del modello}
\fancyhead{}    % reset header
\fancyhead[R]{Retraining del modello}

Una volta inserite le nuove istanze al termine della compilazione del questionario, il modello deve essere \textbf{riallenato} considerando i nuovi valori inseriti.\vspace{1ex}
Per fare ciò è stato necessario creare un \textbf{modulo python} che contenesse un \textbf{interprete} e \textbf{un'ambiente virtuale} in grado di installare tutte le liberie necessarie per l'esecuzione di un file python \textcolor{gray!50!black}{\textit{(trainingModel.py)}} contenente il codice necessario per fare ciò. \\ 

\begin{lstlisting}[caption=Prelievo del dataset]
    # coding: utf-8
    import pandas as pd
    
    FILE_NAME = "../src/main/resources/ModuloAI/data.csv"
    
    # Verifica il contenuto del file
    try:
        df = pd.read_csv(FILE_NAME)
    except pd.errors.EmptyDataError:
        print("Il file CSV e' vuoto.")
    except pd.errors.ParserError:
        print("Errore di parsing del file CSV.")
    except Exception as e:
        print(e)
\end{lstlisting} 

In questa parte si va a prelevare il dataset presente in una cartella specifica del nostro progetto.
Tale dataset è stato estratto in formato \textbf{csv} e tramite il modulo \textbf{pandas} si andrà ad ottenere il \textbf{DataFrame} contente tutte le tuple del dataset.\vspace{1ex}\\
In caso di dataset vuoto viene ritornata un'eccezione (gestita dal metodo chiamante java) come succede anche per errori di \textbf{parsing} del file csv e per errori generici. \\

\begin{lstlisting}[caption=Feature preparation]
   from sklearn.ensemble import RandomForestRegressor
   from sklearn.preprocessing import RobustScaler
   from sklearn2pmml import sklearn2pmml
   from sklearn2pmml.pipeline import PMMLPipeline

   #Scelta variabile dipendente (y) e indipendenti (X)
   X=df.drop(columns=['Duration','VRHeadset',"UserID"])   #feature selection
   y=df.Duration

   # Adattamento e trasformazione dei dati con RobustScaler
   scaler = RobustScaler()
   X_scaled = scaler.fit_transform(X)

  # Creazione del modello di regressione con Random Forest
  model = RandomForestRegressor()
  model.fit(X_scaled, y)
\end{lstlisting} 

In quest altro blocco di codice si va a trasformare il dataset e i dati al suo interno per prepararli al \textbf{training} del modello. ( pragrafo \ref{paragrafo 3.1} per una descrizione più dettagliata). \vspace{1ex}\\ 
Dopo aver applicato il \textbf{metodo empirico} su vari algoritmi e per ogni algortimo aver applicato \textbf{3 tipologie di scaling} diverse, si è visto che l'algoritmo che si adatta meglio ai dati è il \textbf{RandomForestRegressor} applicando però la \textbf{RobustScaler} sui dati.
Per preparare i dati vi sono una serie di passi da seguire:
\par{
\begin{itemize}
   \item \textbf{feature selection}: vengono selezionate le feature \textbf{caratterizzanti} del problema in esame: \textit{Age, Gender, MotionSickness, ImmersionLevel}. Feature come \textit{VRHeadset} non sono state considerate in quanto i visori che verranno utilizzati nelle lezioni virtuali hanno headset diversi dalle 3 tipologie presenti nel dataset. In più, il regressore dovrà stimare la durata ideale di un meeting in una determinata stanza e per fare ciò si basa sul \textit{MotionSickess} che racchiude tutti i fastidi provati durante meeting precedenti, incluso anche quello dell'headset. 
   \item \textbf{feature scaling}: come detto prima, il \textbf{RobustScaler} consente di fare stime più accurate rispetto altri scaler. Da come si legge nel codice, viene applicato lo scaling su X che corrisponde all'insieme di tuple presenti nel dataset su cui è stata applicata la feature selection
   \item \textbf{feature balancing e feature cleaning}: non 
   sono stati applicati per i motivi descritti nel \textit{paragrafo \ref{paragrafo 3.1}}.
\end{itemize}
} \\ 

\begin{lstlisting}[caption=Conversione in un file PMML]
   # Creazione del PMMLPipeline con il modello già addestrato
   pipeline = PMMLPipeline([("Regression", model)])

   # Tentativo di estrazione del pipeline in un file PMML con gestione delle eccezioni
   try:
      sklearn2pmml(pipeline, "RegressoreDurataMeeting.pmml", with_repr=True)
   except Exception as e:
      print("Si è verificato un errore durante l'estrazione del pipeline in un file PMML:")
      print(e)
\end{lstlisting} 

In quest'ultima parte si andrà a convertire il modello già addestrato in un \textbf{file PMML} che verrà utilizzato per compiere stime in un altro modulo di MetaClass (descritto nel paragrafo \ref{paragrafo 5.1}). \vspace{1ex} \\
Per poter compiere la conversione è necessario creare una \textbf{pipeline} contenente il \textbf{modello addestrato}. l'istanza model inoltre conterrà anche tutti i dati del dataset con relativa separazione tra variabili dipendenti e indipendenti. \\
Infine in blocco try/except si creare il file che verrà salvato nella directory corrente del modulo python appena descritto.






\subsection{Creazione dati artificiali}
\fancyhead{}    % reset header
\fancyhead[R]{Creazione dati artificiali}

\newpage
\section{ Glossario }
\fancyhead[]{}
\fancyhead[R]{ Glossario }

\newpage


% End of document
\end{document}


EXAMPLES:

LINK EXAMPLE
\anchor{Name External Link}{https://www.google.com/}

CODE EXAMPLE
\begin{bashCode}{Example Code Block}
    int a = 5;
\end{bashCode}
to insert: %\input{path.tex}

MONOSPACE INLINE TEXT EXAMPLE
\texttt{Example Text}

MONOSPACE TEXT EXAMPLE
\begin{verbatim}
    monospaced text
\end{verbatim}

MATRIX EXAMPLE
$\begin{bmatrix}
    a & b & c\\
    d & f & g
\end{bmatrix}$ · 
$\begin{bmatrix}
    a & b & c\\
    d & f & g
\end{bmatrix}$ = 
$\begin{bmatrix}
    a & b & c\\
    d & f & g
\end{bmatrix}$

IN-LINE EQUATION EXAMPLE
\(x^n + y^n = z^n\)

STAND-ALONE EQUATION EXAMPLE
\[ x^n + y^n = z^n \]

IMAGE EXAMPLE
\begin{figure}[H]
    \centering
    \includegraphics[width=\textwidth]{image.png}
    \caption{caption}
    \label{fig:reference}
\end{figure}

SIDE IMAGES EXAMPLE
\begin{figure}[H]
    \centering
    \begin{subfigure}[b]{0.49\textwidth}
        \centering
        \includegraphics[width=\textwidth]{image1.png}
        \caption{caption1}
        \label{fig:reference1}
    \end{subfigure}
    \hfill
    \begin{subfigure}[b]{0.49\textwidth}
        \centering
        \includegraphics[width=\textwidth]{image2.png}
        \caption{caption2}
        \label{fig:reference2}
    \end{subfigure}
    \caption{total-caption}
    \label{fig:total-reference}
\end{figure}

TABLE EXAMPLE
\begin{table}[!htbp]
    \centering
    \captionsetup{justification=centering}
    \begin{tabular}{|p{4.5cm}|p{9.5cm}|}
        \hline
        a1 & a2\\
        \hline
        b1 & b2 \\
        \hline
    \end{tabular}
    \caption{caption}
    \label{tab:reference}
\end{table}