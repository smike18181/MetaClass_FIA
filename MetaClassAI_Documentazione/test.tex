\documentclass{article}
\usepackage{xcolor}
\usepackage{listings}

% Definisci i colori che desideri utilizzare
\definecolor{codebackground}{rgb}{0.95,0.95,0.95}
\definecolor{codeframe}{rgb}{0.75,0.75,0.75}
\definecolor{codekeyword}{rgb}{0.37,0.08,0.25}

% Impostazioni per il blocco di codice
\lstset{
    language=Python,
    basicstyle=\ttfamily\small,
    keywordstyle=\color{codekeyword}\bfseries,
    backgroundcolor=\color{codebackground},
    frame=single,
    rulecolor=\color{codeframe},
    breaklines=true,
    showstringspaces=false,
    captionpos=b,
}

\begin{document}

\begin{lstlisting}[caption=Esempio di codice Python]
breakable, fontsize=\footnotesize]
    # coding: utf-8
    import pandas as pd
    
    FILE_NAME = "../src/main/resources/ModuloAI/data.csv"
    
    # Verifica il contenuto del file
    try:
        df = pd.read_csv(FILE_NAME)
    except pd.errors.EmptyDataError:
        print("Il file CSV è vuoto.")
    except pd.errors.ParserError:
        print("Errore di parsing del file CSV.")
    except Exception as e:
        print(e)
\end{lstlisting}

\end{document}

